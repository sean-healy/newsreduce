\subsection{Personalised search}
After the development of \pr{} \cite{page1998}, techniques
were suggested by \citeA{page1999} to make search `personalised' to users.
A simple approach to this emerged when \citeauthor{page1999} were
solving the
problem of {\it rank sinks}: webpages with at least one backlink, but
no outgoing links, or pages that form a cyclic clique, where no page
in the clique references a page outside of the clique.  In a naive
\pr{} implementations, these pages would gradually accumulate large
amounts of rank, as they would essentially correspond to final states
in a finite state automaton.  Betting on a user being in a
the final state would always yield large returns.

One solution was
to use a set of predetermined pages as {\it escape routes} from rank
sinks.  In other words, if we consider \pr{} to represent the way a
user might navigate through the web, then the set of escape route pages
may be thought of as the random user's home button, or links on the user's
bookmarks toolbar. By tweaking this set of escape links, and ensuring
they not only accumulate rank from rank sinks, but also from a {\it tax}
applied to all pages, \pr{} can generate rankings relevant to a
given user (or at least to the image of that user formed from their
`bookmark links').

This simple approach is used by \nr{}
(Section \ref{rank}) to calculate webpage ranks from the perspective
of users in different English speaking regions.  This helps ensure that an
Australian reader receives Australian news, an Irish person receives
Irish news, and so on.
