\subsection{Large-scale information retrieval services}
\paragraph{Google} Now ubiquitous, Google search, and the technology behind it, were first introduced by \citeA{page1998}.  Much of the software
engineering techniques outlined in Section \ref{method} draw
from the topics introduced in that paper, including the need
for appropriate compression, distributed web crawling, appropriate word and
URL ID methods, and of course, a ranking algorithm.  \pr{}
is essentially an application of Markov processes and Monte
Carlo simulation, though there is no mention of
either of those topics by \citeA{page1998}.

The task of ranking the importance of webpages, based on a directed graph
made
up of web links, can be explained through an analogy to a statistical
problem in the
game `Monopoly' \cite{euler84}.  In a simple variant of Monopoly, where each tile carries
the same fine (€1), consider the problem: {\it What is the best tile to
purchase}?  The answer, of course, is the tile that people land on most
frequently. To determine which exact tile that is, a Monte Carlo simulation
is initiated, with a probability value of 1 on the start tile, and a
value of 0 on every other tile.  This reflects the fact that before any
moves, a player is bound to be on the start tile.  Next, that probability of 1 is split into 36 pieces (possible dice combinations), and each piece is distributed to the next round of possible player destinations.
The links between tiles are not always trivial, given additional
complicated Monopoly rules.

This probability splitting process is repeated, and over time,
popular destinations accumulate a lot of {\it rank} (probability),
and less popular destinations lose rank.  As the
game progresses, the variance in tile ranks from one move to the next 
begins to converge, and at a certain variance threshold, the
algorithm doesn't evaluate the next move.  The tile ranks are then
finalised.

\pr{} basically works the same way as this example, except that the
tiles are webpages, and the dice rolls are links on the pages.