\clearpage
\section{Wikipedia}

There is a long tradition of researchers mining data from Wikipedia in order to
extract meaning \cite{nguyen2007, nakayama2007, lehmann2015}.  The
hierarchical category structure is particularly useful.
\citeA{nguyen2007} used this hierarchical structure, along with the
first sentence in each article, to extract relations between entities.
Two important {\it relations} mentiond in Section \ref{task} are those
of \underline{{\it is news source page}} and \underline{{\it is homepage}}.  The text
mining approach of \citeA{nguyen2007} will be used by \nr{} to carve out
the collections of entities that satisfy these relations.

Here, {\it relation} and {\it entity} are meant in the predicate
logic sense, where an entity is some object in the world
(real or imaginary): e.g. {\it CNN}, {\it Newspaper}, etc.  A relation is
some factual template that evaluates to true for an object, e.g.
{\it $\rule{1cm}{0.09mm}$ is an author},
{\it $\rule{1cm}{0.09mm}$ is not a news source}.

A popular argument against using Wikipedia as a source of truth
is of course reliability. \citeA{rector2008} found that professional
encyclopedias were indeed more reliable than Wikipedia, in terms
of the number of inaccurate statements per randomly selected topics.  But
the cost of using anything other than Wikipedia at the time of writing
is coverage.  Intuitively, most local newspaper are unlikely to have
an article in Encyclopedia Britannica. The encyclopedia would
need to pay writers and editors from every local area to achieve
that level of coverage.  Wikipedia has high coverage as a result
of worldwide volunteer collaboration.  It is true that most articles
can be edited by anyone, but there are also many quality procedures
in place, and anyone can revert flagrant falsities as easily and as
quickly as they were written.  For very important articles, there
are limits on who can edit.  Finally, \nr{} will also use Pagerank
to choose which sources take precedence (Sections \ref{ir} and \ref{rank}).
For this reason, vandalism on {\it stub articles}\footnote{
{\it Stubs} are short, generally less relevant articles.
\url{https://en.wikipedia.org/wiki/Wikipedia:Stub}
}
leading to false positives or negatives in the two previously mentioned
relations are expected to have lower impact.  For the reasons
outlined, Wikipedia was chosen as the source of truth the questions of
``What is a news source?'' and ``What is the news source's homepage?''