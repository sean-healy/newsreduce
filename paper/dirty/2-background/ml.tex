\section{Machine learning\label{ml-bg}}

Within machine learning, there is currently a distinction between two sets of
algorithms, broadly described as {\it explainable AI} (XAI) and {\it black
box} models. The \nr{} service will benefit from both methods in
some sense; any project using word embeddings has some
unexplainability in it.  The XAI term should first be
disambiguated.

\subsection{Explainable AI} As stated by \citeA{goebel2018}, XAI
is not a new field. The expert systems of the 80s were one
clear example of XAI. The result of a program could be easily
explained by examining the logic rules of the program.
When we speak of XAI, reference is usually being made to the observed
inability of an artificial neural network (ANN)
to {\it explain} its classification decisions. ANNs can display near-human
performance on certain classification tasks \cite{krizhevsky2012}, but the
absence of explanations can make debugging impossible, and bias difficult to
spot. In a system of decision trees, on the other hand, the {\it paper trail}
of a trained program's decision making can be easily extracted by observing
the paths through a decision tree that were taken, and in boosted systems, by
observing the weights at each tree in a forest. The popular example of a situation
where XAI is important emerges in the field of computer vision, and the
racial bias therein. Much research has been published in attempts
to tackle this issue, (e.g. \citeNP{wang2019}). The previously cited paper
of \cite{goebel2018} makes some progress towards XAI by applying
neural networks to the task of assigning textual explanations
to generated image descriptions.

Generally speaking, a relatively {\it explainable} approach
was sought for the main task of this thesis, and that ruled
out ANN models. There are many models that fit the XAI
requirement and still perform well on classification tasks.
An implementations of AdaBoost decision trees, with various
configuration settings, is outlined later in Section \ref{ml}.
The precision and recall of this approach appears sufficient
for the task of news feed personalisation (more results in
Section \ref{results}).  Decision trees allow the user to
gradually build up a model of their news preferences, while
still maintaining the ability to examine the reasons behind
negatively labelled news items.  This offers readers insights
into their underlying preferences, both conscious and
subconscious.

\subsection{Decision trees}

When it comes to natural language text classification, a
single decision tree will generally perform poorly in all
metrics, including precision and recall
(Section \ref{classifier-results}).  But techniques
have been invented to combine many decision trees into a {\it decision
tree forest}.  The final decisions from trees in the forest are
combined in some way to produce results with higher precision and
recall. One such technique is random forest, introduced by \citeA{ho1995}.
and extended by \citeA{breiman2001}.  Another technique is AdaBoost,
first introduced by \cite{schapire1999}.

A decision tree is comparable to the common notion of a
{\it flowchart}, but with no cycles, and each node has exactly one
parent.  This fits closely with the description of a {\it taxonomy}
from biology.  Manually created decision trees are useful for
classifying objects with certain features into different categories.
Beginning at the root of the tree, a question is asked regarding
the object.  The answer to this question determines which branch is
followed next. (Generally, there are two child branches per node.)
A new question is then asked, and the process continues until a
{\it leaf} is reached.  Leaves are the nodes with no child branches,
at the bottom of the tree. Each leaf corresponds to a class.  The
class is then assigned to the object.

\begin{figure}[]
    \centering
    \Tree [.Weekday?
        [.No
            [.Agreed~to~overtime?
                [.No
                    No~work
                ]
                [.Yes
                    Work
                ]
            ]
        ]
        [.Yes
            [.Annual~leave~booked?
                [.No
                    [.Bank~holiday?
                        [.No
                            [.Sick?
                                [.No
                                    [.Family~emergency?
                                        [.No
                                            Work
                                        ]
                                        [.Yes
                                            No~work
                                        ]
                                    ]
                                ]
                                [.Yes
                                    No~work
                                ]
                            ]
                        ]
                        [.Yes
                            No~work
                        ]
                    ]
                ]
                [.Yes
                    No~work
                ]
            ]
        ]
    ]
    \caption{
        An example of a DT people may implicitly use to determine
        if they should go to work each morning.
    }
    \label{dtree}
\end{figure}

Figure \ref{dtree} is a very basic example of a decision tree.
The tree could in fact be built automatically, from a spreadsheet
of data regarding someone's work and life patterns.  That spreadsheet
may look something like Table \ref{dtreed}

\begin{table}
    \begin{tabular}{|l|l|l|l|l|l||l|}
        \hline
        Day of week & Overtime & Annual leave & Bank holiday & Sick & Emergency & Worked\\
        \hline
        {\tt 1} & {\tt F} & {\tt F} & {\tt F} & {\tt F} & {\tt F} & {\tt T}\\
        {\tt 2} & {\tt F} & {\tt F} & {\tt F} & {\tt F} & {\tt F} & {\tt T}\\
        {\tt 3} & {\tt F} & {\tt F} & {\tt F} & {\tt F} & {\tt F} & {\tt T}\\
        {\tt 4} & {\tt F} & {\tt F} & {\tt F} & {\tt F} & {\tt F} & {\tt T}\\
        {\tt 5} & {\tt F} & {\tt F} & {\tt F} & {\tt F} & {\tt F} & {\tt T}\\
        {\tt 6} & {\tt F} & {\tt F} & {\tt F} & {\tt F} & {\tt F} & {\tt F}\\
        {\tt 7} & {\tt F} & {\tt F} & {\tt F} & {\tt F} & {\tt F} & {\tt F}\\
        {\tt 1} & {\tt F} & {\tt T} & {\tt F} & {\tt F} & {\tt F} & {\tt F}\\
        {\tt 2} & {\tt F} & {\tt F} & {\tt T} & {\tt F} & {\tt F} & {\tt F}\\
        {\tt 3} & {\tt F} & {\tt F} & {\tt F} & {\tt T} & {\tt F} & {\tt F}\\
        {\tt 4} & {\tt F} & {\tt F} & {\tt F} & {\tt F} & {\tt T} & {\tt F}\\
        \hline
    \end{tabular}
    \caption{Data used to generate the DT in Figure \ref{dtree}\label{dtreed}}
\end{table}

Some of the data in Table \ref{dtreed} is categorical (boolean), but the
{\it day of the week} variable is numerical, non-continuous.  When implementing
a robust DT learning algorithm, it is important to acccount for both
categorical, numerical and continuous variables.  Otherwise, valuable
features for classification could be excluded from selection.  The CART
algorithm\footnote{Classification and regression trees}
\cite{breiman1984} is one way to achieve this robustness.

\subsection{Choosing the right questions}
DT learning algorithms generally run top-down, first determing the best
question to ask in order to split a dataset into more manageable
parts.  For each part, the same process is applied in order to
determine the best followup questions.  {\it Best} is determined
using various methods, including {\it information gain}
\cite{quinlan1986}, {\it gini impurity} (from CART) and {\it entropy}
(from information theory). Gini impurity is the most straightforward
to calculate:
\begin{equation}
    GI(p, C) = 1 - \sum_{c \in C}p(c)
\end{equation}
$C$ is a set of categories. $p$ returns the ratio of the items that
match the category $c$ over the total number of items.  In English,
Gini impurity is the probability of the following event.  If a random item and
a random category are taken at a DT node, what is the probability
that the category matches the item?  For a node where all items
have the same category, the probability is 1.  The more categories
represented at a node, the lower the probability becomes.  In other
words, using gini impurity to determine the best question to ask
gradually leads to nodes with fewer distinct categories.  This
corresponds closely with the idea of minimising entropy.

\subsection{Random forest}
Random forest \cite{ho1995} applies {\it bagging} to a training data set, and
builds a number of trees from these bags.  Bagging works by
splitting data into random chunks, and applying machine learning
to each chunk separately, in order to build multiple classifiers
(an ensemble).  With random forest, the bags are built by randomly
selecting $N$ items from a dataset of size $N$.  The items are
selected with replacement, meaning that the same item can be
selected more than once.  For each bag, a full decision tree is
built (without pruning).  While building each node in a tree,
a random sampling of $k$ features is taken from the full set of
features, $K$.  There is no required value for $k$, but $k = \sqrt{K}$
is a popular choice.

Trees in a random forest may be built in parallel.  Pruning is not
necessary if many trees are trained in the forest.  Classification
of items involves a vote among all the trees in the forest.  A
threshold is set according to desired precision and recall scores,
and above that threshold, the result is a positive match.
Below the threshold, the result is a negative match (for
a binary classifier).

\subsection{AdaBoost (AB)}
AdaBoost \cite{schapire1999} also generates decision tree forests,
but {\it boosting} is applied to the best performing trees in
the forest, so that their {\it votes} in a classification task
carry more sway than other less performant trees.  The way the
forest is constructed also differs from RF.  In RF, trees can be
built in parallel, because no tree affects the others in the
ensemble.  In AB, trees are built sequentially, since the
performance of the previous tree determines the best questions
to ask in the next tree.