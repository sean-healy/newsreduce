\section{What is news?}
This question is mostly a philosophical one.  But for the practical
purposes of this project, an estimated answer is given by way of examples.
People may agree that things you read in a newspaper are news,
for example {\it The Irish Times}, {\it Reuters}, {\it The Financial
Times}.  But then there are other forms of media that lie in
an ontological sense somewhere between `online newspaper' and
`online radio station', or between `online newspaper' and
`quiz website' ({\it BuzzFeed}), or between `online newspaper' and
`TV station' ({\it BBC}).  For most of the examples given so far, at least
sometimes, these sources contain information people would describe as
news.  In this project, the answer to the question "Is something a
newssource?" is determined by supervised machine learning.  Different
encyclopedia pages are labelled with a binary class, {\it news source} or
{\it not news source}.  From this data, a program is generated that
takes unlabelled encyclopedia pages as input, and returns a probability
score that the encyclopedia page is a {\it news source}.

This pattern of supervised machine learning is repeated for several of the
questions surrounding news aggregation, including ``What is the
official homepage of a news source?'', ``Given a webpage that is an
article, what is the author of the article, the date the article was
published, and what is the headline?  The process is described in
more detail in Section \ref{ml}.