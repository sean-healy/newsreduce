\section{The task\label{task}}
The task of personalised news aggregation is a complex one, but it can
be broken down into several more manageable tasks.  These tasks are
well described by the following questions:

\begin{enumerate}
    \item What is news?
    \item What is a news source?
    \item What is a good/bad news source?
    \item What is the official homepage of a news source?
    \item How do we discern article links on a homepage from other
          links (ads, etc.)?
    \item Who is the author of an article?
    \item When was an article released/published/printed?
    \item Which parts of a webpage actually correspond to article content,
          i.e. headline, byline, body, visual media and captions?
    \item What is a highly importance topic for the current
          news cycle?
    \item How does one identify similar articles, and eliminate all but the best
          sources for that cluster?
    \item Is the {\it best source} the paper that ``got there first'', or
          the paper with highest rank, or the paper that covered the
          topic the best, or some other heuristic based on a
          combination of these metrics?\label{final-q}
\end{enumerate}
\subsection{Motivation}
Some of the 11 questions above may appear trivial.  Arguably, most
people know {\it news} when they see it.  As for the more
complicated problems (e.g. Question \ref{final-q}), a large team
of experts from the world of news could easily create a
pipeline to provide high-quality news aggregation.
But an operation on that scale would cost a lot of money.
Furthermore, the task of news aggregation, i.e. republishing
the work of other writers and journalists, could be considered
very tedious work that ought to be automated.