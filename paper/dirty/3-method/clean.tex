\section{Data cleaning\label{clean}}
\subsection{Alphabet normalisation}
Before constructing prediction models with
the textual data from web pages (Section \ref{training}), the
characters appearing within each page are normalised to
ASCII (a-z).  This reduces the total number of features needed
by the model. The drawback  is a loss of specificity in
languages that rely heavily on accents (French, for example).
That said, as mentioned in Section \ref{task}, this project targets
English language news only.
Alphabet normalisation also reduced the disk space needed to store a
mapping of word IDs to word values.  In order to collect a large
number of candidate characters for normalisation, the names file from
\citeA{unicode} was parsed, and simple string lookups for `LATIN A',
`LATIN B', and so on, were performed.

Figure \ref{alpha} illustrates just a handful of
the different forms latin letters can take.  In most representations
of resources, each of these forms is replaced with the
ASCII letters.  The same process is applied for numbers and punctuation
since there are many ways to quote, hyphenate,
parenthesise, insert pauses into sentences, etc.  All language
representation models have drawbacks, and as stated, the drawback in
this procedure ({\it Unicode equivalence}) is a potential loss
of specificity.  But the larger benefit (for a language like English)
is that commonly anglicised words have a lower chance of being
identified as separate symbols within texts.  A simplified example: if
a news article about {\it Tomáš Mikolov} instead uses the incorrect
form, {\it \underline{Tomas} Mikolov}, after alphabet normalisation,
the article would still identify {\it {\bf Tomáš} Mikolov} as a topic.

Alphabet normalisation is only applied on internal representations,
and the output that news readers see must always be the original
unicode form.

\begin{figure}
    \centering
    \begin{tabular}{|c|l|}
        \hline
        ASCII-form & non-ASCII forms\\
        \hline
        {\tt a} & {\tt A a À Á Â Ã Ä Å à á â ã ä å Ā ā Ă ă Ą ą Ǎ ǎ Ǟ ǟ Ǡ ǡ Ǻ ǻ Ȁ\ldots}\\
        {\tt b} & {\tt B b Ḃ ḃ Ḅ ḅ Ḇ ḇ\ldots}\\
        {\tt c} & {\tt C c Ç ç Ć ć Ĉ ĉ Ċ ċ Č č Ḉ ḉ\ldots}\\
        {\tt d} & {\tt D d Ď ď Đ đ Δ Ḋ ḋ Ḍ ḍ Ḏ ḏ\ldots}\\
        {\tt e} & {\tt E e È É Ê Ë è é ê ë Ē ē Ĕ ĕ Ė ė Ę ę Ě ě Ǝ ǝ Ȅ ȅ Ȇ ȇ Ε\ldots}\\
        {\tt f} & {\tt F f ƒ Φ Ḟ ḟ\ldots}\\
        {\tt g} & {\tt G g Ĝ ĝ Ğ ğ Ġ ġ Ģ ģ Ǧ ǧ Ǵ ǵ Ḡ ḡ\ldots}\\
        {\tt h} & {\tt H h Ĥ ĥ Ħ ħ Ȟ ȟ Ḣ ḣ Ḥ ḥ Ḧ ḧ Ḫ ḫ ẖ\ldots}\\
        {\tt i} & {\tt I i Ì Í Î Ï ì í î ï Ĩ ĩ Ī ī Ĭ ĭ Į į İ ı Ǐ ǐ Ȉ ȉ Ȋ ȋ Ι\ldots}\\
        {\tt j} & {\tt J j Ĵ ĵ ǰ ȷ\ldots}\\
        {\tt k} & {\tt K k Ķ ķ Ǩ ǩ Κ Ḱ ḱ Ḳ ḳ Ḵ ḵ\ldots}\\
        {\tt l} & {\tt L l Ĺ ĺ Ļ ļ Ľ ľ Ŀ ŀ Ł ł Ḷ ḷ Ḹ ḹ Ḻ ḻ\ldots}\\
        {\tt m} & {\tt M m Μ Ḿ ḿ Ṁ ṁ Ṃ ṃ\ldots}\\
        {\tt n} & {\tt N n Ñ ñ Ń ń Ņ ņ Ň ň Ǹ ǹ Ν Ṅ ṅ Ṇ ṇ Ṉ ṉ\ldots}\\
        {\tt o} & {\tt O o Ò Ó Ô Õ Ö Ø ò ó ô õ ö ø Ō ō Ŏ ŏ Ő ő Ơ ơ Ǒ ǒ Ǫ ǫ Ǭ ǭ Ǿ\ldots}\\
        {\tt p} & {\tt P p Π Ṕ ṕ Ṗ ṗ\ldots}\\
        {\tt r} & {\tt R r Ŕ ŕ Ŗ ŗ Ř ř Ȑ ȑ Ȓ ȓ\ldots}\\
        {\tt s} & {\tt S s ß Ś ś Ŝ ŝ Ş ş Š š Σ Ṡ ṡ Ṣ ṣ Ṥ ṥ Ṧ ṧ Ṩ ṩ\ldots}\\
        {\tt t} & {\tt T t Ţ ţ Ť ť Ț ț Τ Ṫ ṫ Ṭ ṭ Ṯ ṯ ẗ}\\
        {\tt u} & {\tt U u Ù Ú Û Ü ù ú û ü Ũ ũ Ū ū Ŭ ŭ Ů ů Ű ű Ų ų Ư ư Ǔ ǔ Ǖ ǖ Ǘ\ldots}\\
        {\tt v} & {\tt V v Ṽ ṽ Ṿ ṿ\ldots}\\
        {\tt w} & {\tt W w Ŵ ŵ Ẁ ẁ Ẃ ẃ Ẅ ẅ Ẇ ẇ Ẉ ẉ ẘ\ldots}\\
        {\tt x} & {\tt X x Ξ Ẋ ẋ Ẍ ẍ\ldots}\\
        {\tt y} & {\tt Y y Ý ý ÿ Ŷ ŷ Ÿ Ȳ ȳ Ẏ ẏ ẙ Ỳ ỳ Ỵ ỵ Ỷ ỷ Ỹ ỹ\ldots}\\
        {\tt z} & {\tt Z z Ź ź Ż ż Ž ž Ζ Ẑ ẑ Ẓ ẓ Ẕ ẕ\ldots}\\
        \hline
    \end{tabular}
    \caption{
        A few of the different letters that can be normalised to ASCII.
    }
    \label{alpha}
\end{figure}


\subsection{Tokenisation}