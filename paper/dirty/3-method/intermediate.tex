\section{Intermediate representations\label{repr}}
As mentioned in Section \ref{crawling}, crawled resources are stored
in their entirety as HTML documents, and the HTTP headers are
stored in separate plaintext files.  For clarity, each
resource will have separate HTML and header files for each
time point at which the resource was crawled, which makes
\nr{} a version archiving program.

But these initial resource formats are a small fraction of the
resource representations stored.  To compare the performance of
machine learning models across different representations,
many representations are generated. Below is a non-exhaustive
list of intermediate formats:
\begin{itemize}
    \item A sub-documents file
    \item A document vector
    \item A links file
    \item A tokens file
    \item A tokens file, with stop words removed
    \item A HTML title file
    \item The following formats are generated separately from the
          two token files, with and without stop words.
    \begin{itemize}
        \item Bag of words (BOW)
        \item Binary bag of words (BBOW)
        \item Binary bag of n-grams
        \begin{itemize}
            \item Binary bag of bigrams (BBOBG)
            \item Binary bag of trigrams (BBOTG)
            \item Binary bag of skip-grams, with skips
                  $\le 2$, bag-size $= 2$ (BBOSG)
        \end{itemize}
    \end{itemize}
\end{itemize}
\subsection{The sub-documents file\label{subd}} This file
is designed specifically for extracting a particular piece of
information (a relation)
from a webpage.  The format extracts all the leaf nodes from a HTML
document, and associates the attributes (text, {\tt href}, title,
etc.) with
the path of the leaf node.  The path is made up of the HTML class
names, IDs, and tag names, in reverse order of tree height
in the document DOM. In a fictitiously simplified webpage, the
sub-document file might look
like this:
\begin{verbatim}
{"text":"Seán Healy"} span.author p div
{"text":"My Homepage"} h1#headline
{"text":"Home", "href"="https://seanh.sh/} a header
\end{verbatim}
Each row in the sub-documents file can be considered a document
in itself.  Extracting relations may then be accomplished with
standard document classification techniques, such as Naive Bayes,
decision trees, etc.