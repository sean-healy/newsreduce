\section{Scheduling}

\citeA{mcbryan1994} aimed to ``tame the web'' in
a general sense, and many others have done the same since then.
But \nr{} is only interested in news data, specifically
recent news data.  For this reason, a directed crawl of the web
was desired, and a custom scheduler service was set up to
accomplish this.

At regular intervals, the service schedules URLs to be crawled via
custom {\tt SQL} queries run against the main database.  The
queries only retrieve URLs for the following criteria:

\begin{enumerate}
    \item Wikipedia category pages that are listed under other Wikipedia
          category pages.
    \item Wikipedia pages that are listed under Wikipedia category pages.
    \item URLs marked as news source homepages.
    \item Same-origin links on a news source homepage.
    \item Pages that are marked as news indexes.
    \item Same-origin links on a news index.
\end{enumerate}

The web pages to be crawled are inserted into a Redis
{\it sorted set}.  This set is polled in parallel by several
crawler machines.